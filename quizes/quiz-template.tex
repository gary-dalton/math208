\documentclass[12pt,letterpaper]{article}
\usepackage[lmargin=1in,rmargin=1in,tmargin=1in,bmargin=1in]{geometry}
\usepackage{amsmath}
\usepackage{amssymb}
\usepackage{graphicx}
%\usepackage{tikz}
%\usetikzlibrary{trees}
\usepackage[shortlabels]{enumitem}
\newcommand*{\Scale}[2][4]{\scalebox{#1}{\ensuremath{#2}}}
% \usepackage{showframe}
\graphicspath{ {./} }
\usepackage{multicol}
\usepackage{multirow}


% -------------------
% Font
% -------------------
\usepackage[T1]{fontenc}
\usepackage{charter}


% -------------------
% Commands
% -------------------
\newcommand{\quiz}[2]{\noindent\textbf{Name: }\makebox[8cm]{\hrulefill} \hfill \textbf{MATH 208} \\  \textbf{Date: #2} \hfill \textbf{Quiz #1}\\}

\newcommand{\prob}{\noindent\textbf{Problem. }}
\newcounter{problem}
\newcommand{\problem}{
	\stepcounter{problem}%
	\noindent \textbf{Problem \theproblem. }%
}
\newcommand{\pspace}{\par\vspace{\baselineskip}}
\newcommand{\ds}{\displaystyle}


% -------------------
% Header & Footer
% -------------------
\usepackage{fancyhdr}

\fancypagestyle{pages}{
	%Headers
	\fancyhead[L]{}
	\fancyhead[C]{}
	\fancyhead[R]{}
\renewcommand{\headrulewidth}{0pt}
	%Footers
	\fancyfoot[L]{}
	\fancyfoot[C]{}
	\fancyfoot[R]{}
\renewcommand{\footrulewidth}{0.0pt}
}
\headheight=0pt
\footskip=14pt

\pagestyle{pages}


% -------------------
% Content
% -------------------
\begin{document}
\quiz{\#0}{DD/MM}


% Question
\prob This is an unnumbered problem. \pspace


% Question 1
\problem This is a numbered problem. \vspace{1.5cm}


% Question 2
\problem This is the second numbered problem. \vfill


% Question 3
\problem This problem has several parts:
	\begin{enumerate}[(a)]
	\item The first part.
	\item The second part. 
	\item The third part. 
	\end{enumerate} \vspace{6cm}



\newpage



% Question 4
\problem Compute the following integral
	\[
	\int e^{-x^2} \;dx
	\] \vspace{3cm}


% Question 5
\problem Explain why everyone loves Mathematics. \pspace


% Question 6
\problem Compute the integral $\ds \int \sin x^2 \;dx$. \vspace{2cm}


% Question 7
\problem In the space provided, prove the Riemann Hypothesis.



\end{document}