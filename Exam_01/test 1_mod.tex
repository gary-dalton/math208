
\documentclass[11pt]{article} % use larger type; default would be 10pt
\usepackage{amsmath,amsfonts,amssymb, amsthm}
\usepackage[utf8]{inputenc} % set input encoding (not needed with XeLaTeX)
\usepackage{mathptmx, mathtools}
\DeclareMathAlphabet{\pazocal}{OMS}{zplm}{m}{n}
\newcommand\inner[2]{\langle #1, #2 \rangle}
\parindent 0em
\parskip .5em
\setlength{\textwidth}{6in} \setlength{\topmargin}{-0.2in}
\setlength{\textheight}{9.0in} \setlength{\oddsidemargin}{0in}
\renewcommand{\baselinestretch}{1.1}
%\setenumerate[0]{label=(\Alph*)}
\def\R{{\mathbb R}}
\def\L{\pazocal{L}}
\DeclarePairedDelimiter\ceil{\lceil}{\rceil}
\DeclarePairedDelimiter\floor{\lfloor}{\rfloor}
%%% Examples of Article customizations
% These packages are optional, depending whether you want the features they provide.
% See the LaTeX Companion or other references for full information.

%%% PAGE DIMENSIONS
\usepackage{geometry} % to change the page dimensions
\geometry{a4paper} % or letterpaper (US) or a5paper or....
% \geometry{margin=2in} % for example, change the margins to 2 inches all round
% \geometry{landscape} % set up the page for landscape
%   read geometry.pdf for detailed page layout information

\usepackage{graphicx} % support the \includegraphics command and options

% \usepackage[parfill]{parskip} % Activate to begin paragraphs with an empty line rather than an indent

%%% PACKAGES
\usepackage{booktabs} % for much better looking tables
\usepackage{array} % for better arrays (eg matrices) in maths
\usepackage{paralist} % very flexible & customisable lists (eg. enumerate/itemize, etc.)
\usepackage{verbatim} % adds environment for commenting out blocks of text & for better verbatim
\usepackage{subfig} % make it possible to include more than one captioned figure/table in a single float
% These packages are all incorporated in the memoir class to one degree or another...

%%% HEADERS & FOOTERS
\usepackage{fancyhdr} % This should be set AFTER setting up the page geometry
\pagestyle{fancy} % options: empty , plain , fancy
\renewcommand{\headrulewidth}{0pt} % customise the layout...
\lhead{}\chead{}\rhead{}
\lfoot{}\cfoot{\thepage}\rfoot{}

%%% SECTION TITLE APPEARANCE
\usepackage{sectsty}
\usepackage{tikz}
%\allsectionsfont{\sffamily\mdseries\upshape} % (See the fntguide.pdf for font help)
% (This matches ConTeXt defaults)

%%% ToC (table of contents) APPEARANCE
\usepackage[nottoc,notlof,notlot]{tocbibind} % Put the bibliography in the ToC
\usepackage[titles,subfigure]{tocloft} % Alter the style of the Table of Contents
\renewcommand{\cftsecfont}{\rmfamily\mdseries\upshape}
\renewcommand{\cftsecpagefont}{\rmfamily\mdseries\upshape} % No bold!
\newtheorem*{lem}{Lemma}

%%% END Article customizations

%%% The "real" document content comes below...

\title{Math 208 Exam 1}
%\author{Liam Jemison}
\date{} % Activate to display a given date or no date (if empty),
         % otherwise the current date is printed 

\begin{document}

\maketitle
\begin{enumerate}
\item Suppose $\$22,000$ is invested at an annual interest rate of $2.75\%.$ Compute the balance after 3 years if interest is compounded monthly.
\item Suppose $\$5,500$ is invested in a retirement account. If the money is invested at a $6\%$ annual rate, compounded continuously, how much will be in the account after $20$ years? 
\item Suppose $\$18,000$ is invested at an annual interest rate of $3.5\%.$ Compute the balance after 10 years if interest is compounded daily.
\item How much do you need to deposit in a bank account that earns interest at a $3\%$ annual rate, compounded quarterly, to have $\$63,500$ in $5$ years?
\item Find the effective annual rate of an investment account that earns $3.5\%$ annual interest compounded quarterly. Round your answer the nearest hundredth of a percent.  
\item Find the effective annual rate of an investment account that earns $3.4\%$ annual interest compounded continuously. Round your answer the nearest hundredth of a percent.  

\item Compute the following, if possible. 
%\text{-} gives a rather hacky shorter minus for matrix alignment purposes
$$\begin{bmatrix}
1 & 2 & 3 \\
\text{-}3 & 2 & \text{-}1
\end{bmatrix}
+ 
\begin{bmatrix}
2 & 5 & \text{-}1\\
4 & \text{-}3 & 2
\end{bmatrix}.
$$
\item Given $A = \begin{bmatrix}
\text{-}1 & 1 \\
6 & 3
\end{bmatrix},$ and $B = \begin{bmatrix}
2 & \text{-}1 \\
8 & 3 \\
\text{-}5 & 2
\end{bmatrix}, $ compute $A - 6B$, if possible.
\item Compute, if possible, the product 
$$\begin{bmatrix}
\text{-}2 & 1 & 0\\
0 & \text{-}1 & 4 
\end{bmatrix}\begin{bmatrix}
4 & 5 \\
\text{-}6 & 3 \\
9 & 0
\end{bmatrix}.$$
\item Find $x,y$ and $z$ that satisfy 
$$\begin{bmatrix}
	2x - y & z \\
	4 & x + 3y
\end{bmatrix}
= 
\begin{bmatrix}
	9 & -8 \\
	-4 & 1
\end{bmatrix}.
$$
\item If $B = \begin{bmatrix} 
10 & 0 \\
0 & 10
\end{bmatrix}$ and $I=\begin{bmatrix} 
1 & 0 \\
0 & 1
\end{bmatrix}$
compute $B*I^{7}.$

\item Use the general approach to inverting square matrices (i.e. not the explicit formula for $2\times2$ matrices) to find the inverse of $$\begin{bmatrix} \text{-}3 & 2 \\ 2 & 4\end{bmatrix}.$$

\item Compute, if possible, the product 
$$\begin{bmatrix}
	\text{-}6 & 6 & 0 \\
	12 & \text{-}4 & 8 \\
	9  & 0& 1
\end{bmatrix}\begin{bmatrix}
1 & 0 & 0\\
0 & 1 & 0 \\
0 & 0 & 1
\end{bmatrix}.$$

\item Find the determinant of the matrix $\begin{bmatrix} 4 &\text{-}20 \\ 2 &\text{-}9\end{bmatrix}$

\item Find $x$ so that the matrix $\begin{bmatrix} 3 & 1 \\ 4x -1 & 4x + 3\end{bmatrix}$ is not invertible. 

\item Let $A = \begin{bmatrix} 5 &8\\ \text{-}7 &\text{-}11\end{bmatrix}.$
	\begin{enumerate}
		\item[(a)] Find $A^{-1},$ using the formula for the inverse of a $2\times 2$ matrix.  
		\item[(b)] Find $A \cdot A^{-1}.$
	\end{enumerate}



\item Find $x_1$ and $x_2$ that satisfy $$\begin{bmatrix} x_1 \\ x_2 \end{bmatrix} = \begin{bmatrix} 3 & \text{-}2 \\ 1 & 4 \end{bmatrix} \cdot \begin{bmatrix} \text{-}2 \\ 1 \end{bmatrix}.$$
\item Find $x_1$ and $x_2$ using the inverse of the coefficient matrix in the equation 
$$\begin{bmatrix} 1 & 3 \\ 1 & 4 \end{bmatrix} \cdot \begin{bmatrix} x_1 \\ x_2 \end{bmatrix} =  \begin{bmatrix} 9 \\ 6 \end{bmatrix}.$$

\item Explain why the following system of linear equations cannot be solved using the inverse of the coefficient matrix. 
\begin{align*}
-3x_1 + 4x_2 &= 7 \\
6x_1 - 8x_2 &= -14
\end{align*}
\item Find a solution, if one exists, to the sytem of linear equations in problem $19.$
\end{enumerate}
\section*{Formulas}
\begin{align*}
A = &P(1+rt)  \\
	A = &P(1+i)^n  \\
	A = &P\left(1+\dfrac{r}{m}\right)^{mt}  \\
	A = &Pe^{rt} \\
	APY &= (1+r/m)^m -1 \\
	APY &= e^r -1 \\
	&A = \text{ Amount / Future value} \\
	&P = \text{ Principal / Present value} \\
	&i = \text{ decimal rate per compounding period} \\
	&n = \text{ total number of compounding periods} \\
	&r = \text{ decimal annual nominal interest rate} \\
	&m = \text{ number of compounding periods per year} \\
	&t = \text{ time in years} \\
	&\text{Note that } i=r/m \text{ and } n= mt \\
	&APY = \text{Annual Percentage Yield}
\end{align*}
Consider the matrix $A = \begin{bmatrix} a & b \\ c & d \end{bmatrix}.$ Then the inverse of $A$ is 
$$\begin{bmatrix} a & b \\ c & d \end{bmatrix}^{-1} = \dfrac{1}{D}\begin{bmatrix} d & \text{-}b \\ \text{-}c & a \end{bmatrix},$$ as long as $D,$ the determinant of $A,$ given by $D = ad-bc$ is nonzero.
\end{document}