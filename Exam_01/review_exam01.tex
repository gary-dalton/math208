\documentclass[12pt]{article}
\usepackage{amsmath}
\usepackage{amssymb}
\usepackage[top=1in, bottom=0.75in, left=0.75in, right=0.75in]{geometry}
\newcommand*{\Scale}[2][4]{\scalebox{#1}{\ensuremath{#2}}}%
% \usepackage{showframe}
\usepackage{multicol}
\setlength{\columnsep}{1cm}


\begin{document}
\title{\vspace{-10ex}Math 208: Review for Chapters 3 and 4}
\date{\vspace{-10ex}}
\maketitle

\section*{Chapter 3}
\begin{multicols}{2}
	[Some Formula]
	\begin{itemize}
		\item $I=Prt$
		\item $A=P(1+rt)$
		\item $A = P(1+ \frac{r}{m})^{mt}$
		\item $A = Pe^{rt}$
		\item $APY = (1 + \frac{r}{m})^m -1$
		\item $APY = e^r -1$
	\end{itemize}
\end{multicols}

\subsection*{Sample Problems from Chapter 3}
There will be 6 questions from this section on the exam.
\begin{itemize}
\item \textbf{3.1: Problem 87}: In Alabama, finance charges on a payday loan may not exceed 17.5\% of the amount advanced. Find the annual interest rate if \$500 is borrowed for 10 days at the maximum allowable charge.
\item \textbf{3.2 Problem 41}: If \$8,000 is invested at 7\% compounded continuously, what is the amount after 6 years?
\item \textbf{3.2 Problem 65}: A newborn child receives a \$20,000 gift toward college from her grandparents. How much will the \$20,000 be worth in 17 years if it is invested at 7\% compounded quarterly?
\item \textbf{3.2 Problem 77}:  An Individual Retirement Account (IRA) has \$20,000 in it, and the owner decides not to add any more money to the account other than interest earned at 6\% compounded daily. How much will be in the account 35 years from now when the owner reaches retirement age?
\item \textbf{3.2 Problem 53}: What is the effective rate (APY) for money invested at an annual rate of
(A) 5.15\% compounded continuously? and (B) 5.20\% compounded semiannually?
\item \textbf{3.2 Problem 79}: How long will it take money to double if it is invested at 7\% compounded daily? 8.2\% compounded continuously?


\end{itemize}


\section*{Chapter 4}
\subsection*{Formulas  and Concepts Needed}
Given $A = \begin{bmatrix}
a & b \\
c & d
\end{bmatrix}$, $I = \begin{bmatrix}
1 & 0 \\
0 & 1
\end{bmatrix}$, $X = \begin{bmatrix}
x_1 \\
x_2
\end{bmatrix}$, and $B = \begin{bmatrix}
b_1 \\
b_2
\end{bmatrix}
$

\begin{itemize}
\item Determinant: $det(A) = a*d - b*c$
\item $A^{-1} = 1/det(A) * \begin{bmatrix}
d & -b \\
-c & a
\end{bmatrix}$
\item $A^{-1}$ does not exist if $det(A)=0$ and \textbf{A is singular}
\item Recall the Basic Properties of Matrices from page 236
\item $A * A^{-1} = I$
\item $A * I = A = I*A$
\item $A*X = B$
\item $X = A^{-1}*B$
\item Apply Matrix Multiplication and Addition (know when these won't work)
\item Apply augmented matrix for solution to system of equations and $A^{-1}$
\item Apply row reduction methods to find solution to system of equations and $A^{-1}$
\item Recognize and understand \textit{unique solution, infinite solutions}, and \textit{no solution}
\item Recall and apply \textit{solving by substitution} and \textit{solving by elimination by addition} from Chapter 4.1
\end{itemize}

\subsection*{Sample Problems from Chapter 4}
There will be 14 questions from this section on the exam.
\begin{itemize}
\item \textbf{Solve using substitution and addition method}: 
\begin{align*} 
3x - 5y &= 4 \\ 
-9x + 15y &= 12
\end{align*}

\item \textbf{Find x, y, and z}:
Given
$$\begin{bmatrix}
-x + 5y & z \\
-3 & 3x + 2y
\end{bmatrix}
= 
\begin{bmatrix}
2 & -9 \\
-3 & 11
\end{bmatrix}
$$

\item \textbf{Page 221 \#37 and \#43}
\item \textbf{Page 233 \#17}
\item \textbf{Page 234 \#37}: Explain why it is singular.
\item \textbf{Page 234 \#53}: Find the inverse using both the determinant method and using row operations.
\item \textbf{Page 242 \#17 and \#23}

\end{itemize}

\end{document}
