\documentclass[14pt]{extarticle}
\usepackage{amsmath}
\usepackage{amssymb}
\usepackage{graphicx}
\usepackage{tikz}
\usetikzlibrary{calc}
%\usetikzlibrary{trees}
\usepackage{graphicx}
\graphicspath{ {../../chap04/} }
\usepackage[top=0.75in, bottom=0.75in, left=0.75in, right=0.75in]{geometry}
\newcommand*{\Scale}[2][4]{\scalebox{#1}{\ensuremath{#2}}}%
\usepackage[shortlabels]{enumitem}
% \usepackage{showframe}
\title{\vspace{-5ex}Math 208 Section 4.5}
\date{\vspace{-10ex}}
\usepackage{multicol}
\setlength{\columnsep}{1cm}
\newcommand{\tikzmark}[1]{\tikz[overlay,remember picture] \node (#1) {};}
\newcommand{\DrawBox}[4][]{%
	\tikz[overlay,remember picture]{%
		\coordinate (TopLeft)     at ($(#2)+(-0.2em,0.9em)$);
		\coordinate (BottomRight) at ($(#3)+(0.2em,-0.3em)$);
		%
		\path (TopLeft); \pgfgetlastxy{\XCoord}{\IgnoreCoord};
		\path (BottomRight); \pgfgetlastxy{\IgnoreCoord}{\YCoord};
		\coordinate (LabelPoint) at ($(\XCoord,\YCoord)!0.5!(BottomRight)$);
		%
		\draw [red,#1] (TopLeft) rectangle (BottomRight);
		\node [below, #1, fill=none, fill opacity=1] at (LabelPoint) {#4};
	}
}

\begin{document}
\maketitle		
\section*{Homework, Reading, and Other}
\begin{itemize}
	\item Section 4.4
	\item Section 4.5
	\item Section 4.6
\end{itemize}

\section*{Quiz 2}

\section*{Goals}
\begin{itemize}
	\item Describe, understand, and determine the Identity Matrix.
	\item Understand a multiplicative Inverse and recognize that $M*M^{-1} = I$.
	\item State basic properties of a determinant and find the determinate of a $2 \times 2$ matrix.
	\item Find the Inverse of a matrix using Cramer's Rule and by Gauss-Jordan.
\end{itemize}

\section*{4.5 Inverse of a Square Matrix}
Thus far, we have learned what a matrix is and how to add, subtract, and multiply matrices by other matrices or by scalars. We only need to add a few more concepts and we can almost work with matrices as we do with other algebraic elements. In fact, there is a branch of mathematics, called \textit{Linear Algebra}, that is devoted to this study.

\subsection{Identity Matrix}
The \textit{Identity Matrix} is $1$ in matrix math. It is a square matrix ($n\times n$), with $1$s on the diagonal and $0$s elsewhere. We denote it by $I_n$ or simply $I$.
\begin{align*}
	I = I_n = \begin{bmatrix}
		1 & 0 & 0 & \cdots & 0 \\
		0 & 1 & 0 & \cdots & 0 \\
		0 & 0 & 1 & \cdots & 0 \\
		\vdots & \vdots & \vdots & \ddots & \vdots \\
		0 & 0 & 0 &\cdots & 1
	\end{bmatrix} = 
	\begin{bmatrix}
		1 &  &  &  & \\
		 & 1 &  &  &  \\
		 &  & 1 &  &  \\
		 &  &  & \ddots &  \\
		 &  &  & & 1
	\end{bmatrix}	
\end{align*}
Recall that any number multiplied by 1 equals the original number. The same is true for matrices. So that $AI=A = IA$, notice this operation is commutative.
\\\\
Find $AI$ and $IB$ using the rules of matrix products, given
\begin{align*}
	&A = \begin{bmatrix}
		5 & 6 \\
		7 & 8
	\end{bmatrix}
	&B = \begin{bmatrix}
		3 & 2 & 1 \\
		1 & 2 & 3 \\
		2 & 3 & 1
	\end{bmatrix}
\end{align*}

\subsection{Mulitplicative Inverse}
Any number multiplied by its \textit{Mulitplicative Inverse} is equal to 1. Generally, inverse is taken to mean mulitplicative inverse and not the additive inverse. For rational numbers this looks like:
\begin{align*}
	2 \times 1/2 &= 2 \times 2^{-1} = 1 &\text{The inverse of $2$ is $1/2$ and the inverse of $1/2$ is $2$} \\
	98 \times 1/98 &= 98 \times 98^{-1} =1 &\text{The inverse of $98$ is $1/98$ and the inverse of $1/98$ is $98$}
\end{align*}
Some but not all matrices have an inverse. If a matrix does not have an inverse, we call it \textit{singular}. The inverse of a matrix, $A$, is given by $A^{-1}$ , such that
\begin{align*}
	AA^{-1} = I = A^{-1}A
\end{align*}

How do we find the inverse of a matrix? There are two methods. For a $2\times 2$ matrix, we may use the determinant of a matrix and apply Cramer's rule. For any size square matrix, we use an augmented matrix and Gauss-Jordan to find the inverse.

\subsection{Determinant}
The \textit{determinant} of a square matrix is a single number that reveals a lot about the matrix. The determinant is 0 when the matrix has no inverse, i.e. the matrix is singular. Determinants may be calculated for any square matrix but in this class we limit it to a $2\times 2$ matrix.
\begin{align*}
	\text{Given } &M = \begin{bmatrix}
		a & b \\
		c & d
	\end{bmatrix} \\
	\text{then } &\det(M) = ad-bc
\end{align*}
Remember, for a singular matrix, $M$, the $\det(M)=0$.

\subsubsection{Example}
\begin{itemize}
	\item Find $x$ such that $A$ is singular given $A=\begin{bmatrix} 1 & x \\ 2 & 4	\end{bmatrix}$.
	\\\\
	$A$ is singular when its determinant is $0$, thus
\begin{align*}
	\det(A) = 1(4) - 2x &= 0 \\
	x &= 2
\end{align*}
\end{itemize}

\subsection{Cramer's Rule}
The inverse of any $2\times 2$ non-singular matrix, M, may be calculated using \textit{Cramer's Rule}.
\begin{align*}
	\text{Given } M &= \begin{bmatrix}
		a & b \\
		c & d
	\end{bmatrix} \\
	M^{-1} &= \frac{1}{\det(M)} \begin{bmatrix}
		d & -b \\
		-c & a
	\end{bmatrix} = \frac{1}{ad-bc} \begin{bmatrix}
	d & -b \\
	-c & a
\end{bmatrix}
\end{align*}

\subsubsection{Example}
\begin{itemize}
	\item Find $A^{-1}$ given $A=\begin{bmatrix} 1 & -3 \\ 2 & -4	\end{bmatrix}$.
	\begin{align*}
		\det(A) = 1(-4) - 2(-3) &= 2
	\end{align*}
	\begin{align*}
		A^{-1} &= \frac{1}{2}\begin{bmatrix} -4 & 3 \\ -2 & 1	\end{bmatrix}
		= \begin{bmatrix} -2 & 3/2 \\ -1 & 1/2	\end{bmatrix}
	\end{align*}
\end{itemize}

\subsection{Inverse by Gauss-Jordan}
Cramer's Rules does not work for matrices larger than $2\times 2$, so we turn to the Gauss-Jordan method that we learned previously only we start with a different augmented matrix.

Augment the given matrix, M, with I.
\begin{align*}
	\text{Given } M &= \begin{bmatrix}
		a & b \\
		c & d
	\end{bmatrix} \\
	\text{The augmented matrix is }  &\begin{bmatrix}
		a & b & | & 1 & 0 \\
		c & d & | & 0 & 1
	\end{bmatrix}
\end{align*}

To find the inverse, use Gauss-Jordan to get the reduced form
\begin{align*}
	\begin{bmatrix}
		1 & 0 & | & p & q \\
		0 & 1 & | & r & s
	\end{bmatrix} \\
	\text{Then } M^{-1} = \begin{bmatrix}
		p & q \\
		q & s
	\end{bmatrix}
\end{align*}
This method works for any non-singular square matrix. If the matrix is singular, you will get a row of zeros instead of I on the left of the augmented matrix.

\subsubsection{Example}
\begin{itemize}
	\item Find $A^{-1}$ given $A=\begin{bmatrix} 1 & -3 \\ 2 & -4	\end{bmatrix}$.
	\begin{align*}
		\begin{bmatrix}
			1 & -3 & | & 1 & 0 \\
			2 & -4 & | & 0 & 1
		\end{bmatrix} &\to
		\begin{array}{r}
			R1 \\
			-2R1 + R2
		\end{array}
		\begin{bmatrix}
			1 & -3 & | & 1 & 0 \\
			0 & 2 & | & -2 & 1
		\end{bmatrix} \to \\
		\begin{array}{r}
			R1 \\
			R2/2
		\end{array}
		\begin{bmatrix}
			1 & -3 & | & 1 & 0 \\
			0 & 1 & | & -1 & 1/2
		\end{bmatrix} &\to
		\begin{array}{r}
			3R2+R1 \\
			R2
		\end{array}
		\begin{bmatrix}
			1 & -3 & | & -2 & 3/2 \\
			0 & 1 & | & -1 & 1/2
		\end{bmatrix}
	\end{align*}
	\begin{align*}
		A^{-1} &= \begin{bmatrix} -2 & 3/2 \\ -1 & 1/2	\end{bmatrix}
	\end{align*}
\end{itemize}

\noindent\rule{\textwidth}{1pt}
{\footnotesize Copyright (C) 2021 Garold Dalton --- Released under GNU General Public License v3.0}


\end{document}
