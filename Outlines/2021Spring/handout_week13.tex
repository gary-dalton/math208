\documentclass[14pt]{extarticle}
\usepackage{amsmath}
\usepackage{amssymb}
\usepackage{nccmath}
\usepackage{tikz}
%\usetikzlibrary{calc}
\usetikzlibrary{trees}
\usepackage{hyperref}
\usepackage{graphicx}
\graphicspath{ {../../chap10/} }
\usepackage[top=0.75in, bottom=0.75in, left=0.75in, right=0.75in]{geometry}
\newcommand*{\Scale}[2][1.5]{\scalebox{#1}{\ensuremath{#2}}}%
\usepackage[shortlabels]{enumitem}
\usepackage[most]{tcolorbox}
\definecolor{bg}{RGB}{255,249,227}
% \usepackage{showframe}
%\usepackage{caption}
%\usepackage{fdsymbol}
\usepackage{multicol}
\setlength{\columnsep}{1cm}



\title{\vspace{-5ex}HANDOUT Math 208 Week 13}
\date{\vspace{-10ex}}

\begin{document}
\maketitle		

\section{Review}
\begin{enumerate}
	\item At what nominal rate compounded continuously must money be invested to double in 10 years?
	\vspace{2.5cm}

	\item $f(x) = \Scale[1.5]{\frac{2x}{e^x}}$
	\vspace{2.5cm}
	
	\item $ f(x) = x^7\ln x$
	\vspace{2.5cm}
	
\end{enumerate}

\section{Chain Rule and Composite Functions}
Round 1: Find the Interior and Exterior composition functions\\
Round 2: Find the derivative
	\begin{enumerate}
		\item $f(x) = 50e^{-2x}$
		\vspace{3cm}
\cleardoublepage		
		\item $f(x) = \sqrt[3]{1+x^3}$
		\vspace{3cm}
		
		\item $f(x) = (5x+2)^3$
		\vspace{3cm}
		
		\item $f(x) = (x^4 - 5)^5$
		\vspace{3cm}
			
		\item $f(x) = \Scale[1.5]{\frac{1}{(x^2+4)^2}}$
		\vspace{3cm}
		
		\item $g(w) = \sqrt{4-w}$
		\vspace{3cm}
		
		\item $f(x) = (2x^3 + 4)^{-5}$
		\vspace{3cm}
		
		\item $f(x) = e^{3x^4+6}$
		\vspace{3cm}
		
		\item $ f(x) = \ln(x^2+9x+4)$	
		\vspace{3cm}
		
		\item $h(x) = (\ln(1+e^x))^3$
		\vspace{3cm}
		
		\item $g(x) =e^{2x}$
		\vspace{3cm}
		
		\item $h(x) = \ln(x^2+9)$
		\vspace{3cm}
		
		\item $f(x) = (1+e^{x^2})^3$
	\end{enumerate}
\vspace{3cm}

\section{Elasticity of Demand}

	\begin{enumerate}
	\item Given $f(t) = 2.7t+282$ What is the percentage rate of change when $t=5$?
	\vspace{3cm}
	
	\item Given $f(x) = 60x- 1.2x^2$, what is the relative rate of change?
	\vspace{3cm}
			
	\item Given $x=f(p)=1000(40-p)$, find $E(p)$ when $p=8, p=30$, and $p=20$. Interpret these price points in terms of elasticity.
	\vspace{3cm}
	
	\item Let $x = f(p) =1875-p^2$. Is demand elastic, inelastic, or unit elasticity at $p=15, p=25$, and $p=40$?
	\vspace{3cm}
	
	\item Using the price-demand equation $p+0.004x = 32$ for $0 \leq p \leq 32$. 
	\begin{enumerate}
		\item Find the elasticity of demand when $p = 28$.
		\item If the \$28 price is decreased by 6\%, what is the approximate percentage change in demand?
		\item Find all values of p for which demand is elastic.
	\end{enumerate} 
	
\end{enumerate}


\cleardoublepage


\end{document}
