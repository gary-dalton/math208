\documentclass[14pt]{extarticle}
\usepackage{amsmath}
\usepackage{amssymb}
\usepackage{tikz}
\usetikzlibrary{calc}
%\usetikzlibrary{trees}
\usepackage{graphicx}
\graphicspath{ {../../chap04/} }
\usepackage[top=0.75in, bottom=0.75in, left=0.75in, right=0.75in]{geometry}
\newcommand*{\Scale}[2][4]{\scalebox{#1}{\ensuremath{#2}}}%
\usepackage[shortlabels]{enumitem}
% \usepackage{showframe}
\title{\vspace{-5ex}Math 208 Exam 01 Review}
\date{\vspace{-10ex}}
\usepackage{multicol}
\setlength{\columnsep}{1cm}
\newcommand{\tikzmark}[1]{\tikz[overlay,remember picture] \node (#1) {};}
\newcommand{\DrawBox}[4][]{%
	\tikz[overlay,remember picture]{%
		\coordinate (TopLeft)     at ($(#2)+(-0.2em,0.9em)$);
		\coordinate (BottomRight) at ($(#3)+(0.2em,-0.3em)$);
		%
		\path (TopLeft); \pgfgetlastxy{\XCoord}{\IgnoreCoord};
		\path (BottomRight); \pgfgetlastxy{\IgnoreCoord}{\YCoord};
		\coordinate (LabelPoint) at ($(\XCoord,\YCoord)!0.5!(BottomRight)$);
		%
		\draw [red,#1] (TopLeft) rectangle (BottomRight);
		\node [below, #1, fill=none, fill opacity=1] at (LabelPoint) {#4};
	}
}

\begin{document}
\maketitle

\section*{Exam 01 Information}
\begin{itemize}
	\item Thursday, 1/18, during class time.
	\item It will be virtual on Collaborate Ultra session.
	\item See the  \textit{Cover for the Virtual Exam}.
\end{itemize}	

	
\section*{Chapter 3}

\subsection*{Formulas Needed}
\begin{itemize}
	\item $A = P(1+ r/m)^{mt}$
	\item $A = Pe^{rt}$
	\item $APY = (1 + r/m)^m -1$
	\item $APY = e^r -1$
\end{itemize}

\subsection*{Sample Problems from Chapter 3.2}
There will be 6 questions from this section on the exam.
\begin{itemize}
	\item \textbf{Problem 41}: If \$8,000 is invested at 7\% compounded continuously, what is the amount after 6 years?
	\item \textbf{Problem 65}: A newborn child receives a \$20,000 gift toward college from her grandparents. How much will the \$20,000 be worth in 17 years if it is invested at 7\% compounded quarterly?
	\item \textbf{Problem 77}:  An Individual Retirement Account (IRA) has \$20,000 in it, and the owner decides not to add any more money to the account other than interest earned at 6\% compounded daily. How much will be in the account 35 years from now when the owner reaches retirement age?
	\item \textbf{Problem 53}: What is the effective rate (APY) for money invested at an annual rate of
	(A) 5.15\% compounded continuously? and (B) 5.20\% compounded semiannually?
	
\end{itemize}

\section*{Chapter 4}
\subsection*{Formulas  and Concepts Needed}
Given $A = \begin{bmatrix}
	a & b \\
	c & d
\end{bmatrix}$, $I = \begin{bmatrix}
	1 & 0 \\
	0 & 1
\end{bmatrix}$, $X = \begin{bmatrix}
	x_1 \\
	x_2
\end{bmatrix}$, and $B = \begin{bmatrix}
	b_1 \\
	b_2
\end{bmatrix}
$

\begin{itemize}
	\item Determinant: $det(A) = a*d - b*c$
	\item $A^{-1} = 1/det(A) * \begin{bmatrix}
		d & -b \\
		-c & a
	\end{bmatrix}$
	\item $A^{-1}$ does not exist if $det(A)=0$ and \textbf{A is singular}
	\item Recall the Basic Properties of Matrices from page 236
	\item $A * A^{-1} = I$
	\item $A * I = A = I*A$
	\item $A*X = B$
	\item $X = A^{-1}*B$
	\item Apply Matrix Multiplication and Addition (know when these won't work)
	\item Apply augmented matrix for solution to system of equations and $A^{-1}$
	\item Apply row reduction methods to find solution to system of equations and $A^{-1}$
	\item Recognize and understand \textit{unique solution, infinite solutions}, and \textit{no solution}
	\item Recall and apply \textit{solving by substitution} and \textit{solving by elimination by addition} from Chapter 4.1
\end{itemize}	


\subsection*{Sample Problems from Chapter 4}
There will be 14 questions from this section on the exam.
\begin{itemize}
	\item \textbf{Solve using addition method}: 
	\begin{align*} 
		3x - 5y &= 4 \\ 
		-9x + 15y &= 12
	\end{align*}
	
	\item \textbf{Find x, y, and z}:
	Given
	$$\begin{bmatrix}
		-x + 5y & z \\
		-3 & 3x + 2y
	\end{bmatrix}
	= 
	\begin{bmatrix}
		2 & -9 \\
		-3 & 11
	\end{bmatrix}
	$$
	
	\item \textbf{Page 221 \#37 and \#43}
	\item \textbf{Page 233 \#17}
	\item \textbf{Page 234 \#37}: Explain why it is singular.
	\item \textbf{Page 234 \#53}: Find the inverse using both the determinant method and using row operations.
	\item \textbf{Page 242 \#17 and \#23}

\end{itemize}



\cleardoublepage


\end{document}
