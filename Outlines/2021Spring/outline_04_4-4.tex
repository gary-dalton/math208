\documentclass[14pt]{extarticle}
\usepackage{amsmath}
\usepackage{amssymb}
\usepackage{graphicx}
\graphicspath{ {../../chap04/} }
\usepackage[top=0.75in, bottom=0.75in, left=0.75in, right=0.75in]{geometry}
\newcommand*{\Scale}[2][4]{\scalebox{#1}{\ensuremath{#2}}}%
\usepackage[shortlabels]{enumitem}
% \usepackage{showframe}
\title{\vspace{-5ex}Math 208 Section 4.4}
\date{\vspace{-10ex}}
\usepackage{multicol}
\setlength{\columnsep}{1cm}
\usepackage{xcolor}
\newcommand{\yhighlight}[1]{%
	\colorbox{yellow!50}{$\displaystyle#1$}}
\newcommand{\bhighlight}[1]{%
	\colorbox{blue!20}{$\displaystyle#1$}}
\usepackage{tikz}
\usetikzlibrary{calc}
\newcommand{\tikzmark}[1]{\tikz[overlay,remember picture] \node (#1) {};}
\newcommand{\DrawBox}[4][]{%
	\tikz[overlay,remember picture]{%
		\coordinate (TopLeft)     at ($(#2)+(-0.2em,0.9em)$);
		\coordinate (BottomRight) at ($(#3)+(0.2em,-0.3em)$);
		%
		\path (TopLeft); \pgfgetlastxy{\XCoord}{\IgnoreCoord};
		\path (BottomRight); \pgfgetlastxy{\IgnoreCoord}{\YCoord};
		\coordinate (LabelPoint) at ($(\XCoord,\YCoord)!0.5!(BottomRight)$);
		%
		\draw [red,#1] (TopLeft) rectangle (BottomRight);
		\node [below, #1, fill=none, fill opacity=1] at (LabelPoint) {#4};
	}
}

\begin{document}
\maketitle		
\section*{Homework, Reading, and Other}
\begin{itemize}
	\item Section 4.3
	\item Section 4.4
	\item Section 4.5
	\item Quiz 2
\end{itemize}

\section*{Goals}
\begin{itemize}
	\item Add and subtract matrices
	\item Multiply a matrix by a scalar
	\item Calculate the matrix product of a row vector and column vector
	\item Determine if the matrix product of $A_{n\times m} \times B_{p\times r}$ exists and its dimension.
	\item Calculate the matrix product and understand that $A\times B$ is not equivalent to $B\times A$.
\end{itemize}

\section*{4.4 Matrices: Basic Operations}
For all discussions in this section, let $A$ be an $n\times m$ matrix and let $B$ be a $p\times q$ matrix, where $n, p$ are the number of rows and $m, q$ are the number of columns. For any examples below, let:
\begin{align*}
	&A = \begin{bmatrix}
		1 \\
		2 \\
		3
	\end{bmatrix} &
	&B = \begin{bmatrix}
		4 \\
		0 \\
		6
	\end{bmatrix} &
	&C =\begin{bmatrix}
		4 & 5 & 6
	\end{bmatrix} \\
	&D = \begin{bmatrix}
		1 & 2 & 3 \\
		4 & 0 & 6
	\end{bmatrix} &
	&E = \begin{bmatrix}
		2 & 1 \\
		1 & 2 \\
		3 & 4
	\end{bmatrix} &
	&F = \begin{bmatrix}
		3 & 3 \\
		4 & 1 \\
		2 & 0
	\end{bmatrix}
\end{align*}

\subsection{Addition and Subtraction}
Matrices must be of the same dimension to perform addition or subtraction. In other words A and B must have the same number of rows and columns. If they are not the same dimension, then the operation is \textit{not defined}. If the operation is defined, then addition and subtraction are performed elementwise and the result matrix is the same dimension as A and B.
\begin{align*}
	\begin{bmatrix}
		a_{1 1} & a_{1 2} & a_{1 3} \\
		a_{2 1} & a_{2 2} & a_{2 3}
	\end{bmatrix} + 
	\begin{bmatrix}
		b_{1 1} & b_{1 2} & b_{1 3} \\
		b_{2 1} & b_{2 2} & b_{2 3}
	\end{bmatrix} = 
	\begin{bmatrix}
		a_{1 1}+b_{1 1} & a_{1 2}+b_{1 2} & a_{1 3}+b_{1 3} \\
		a_{2 1}+b_{2 1} & a_{2 2}+b_{2 2} & a_{2 3}+b_{2 3}
	\end{bmatrix}
\end{align*}
\subsubsection*{Commutative and Associative}
Let $A,B,C$ be matrices of the same dimensions then
\begin{itemize}
	\item $A+B = B+A$
	\item $(A+B)+C= A+(B+C)$
\end{itemize}
\subsubsection*{Examples}
\begin{itemize}
	\item $A-B$
\begin{align*}
	\begin{bmatrix}
		1 \\
		2 \\
		3
	\end{bmatrix} -
	\begin{bmatrix}
		4 \\
		0 \\
		6
	\end{bmatrix}
	= \begin{bmatrix}
		1 -4\\
		2 -0\\
		3 -6
	\end{bmatrix}
	= \begin{bmatrix}
		-3 \\
		2 \\
		-3
	\end{bmatrix}
\end{align*}
\item $A+C$ is not defined.
\item $D+E$ is not defined.
\item $E+F$
\begin{align*}
	\begin{bmatrix}
		2 & 1 \\
		1 & 2 \\
		3 & 4
	\end{bmatrix} +
	 \begin{bmatrix}
		3 & 3 \\
		4 & 1 \\
		2 & 0
	\end{bmatrix} =
	\begin{bmatrix}
		2+3 & 1+3 \\
		1+4 & 2+1 \\
		3+2 & 4+0
	\end{bmatrix} =
	\begin{bmatrix}
		5 & 4 \\
		5 & 3 \\
		5 & 4
	\end{bmatrix}
\end{align*}
\end{itemize}


\subsection{Scalar Multiplication}
Any matrix may be multiplied by a scalar (a constant number).
\subsubsection*{Examples}
\begin{itemize}
	\item $4*A$
	\begin{align*}
		4 * \begin{bmatrix}
			1 \\
			2 \\
			3
		\end{bmatrix} = 
		\begin{bmatrix}
			4 \\
			8 \\
			12
		\end{bmatrix}
	\end{align*}
	\item $1/2*C$
	\begin{align*}
		-1/2 * \begin{bmatrix}
			4 & 5 & 6
		\end{bmatrix} = 
		\begin{bmatrix}
			-2 & -5/2 & -3
		\end{bmatrix}
	\end{align*}
\end{itemize}

\subsection{Matrix Product of a Row Vector and Column Vector}
Given a $1\times n$ row vector, $A=\begin{bmatrix}a_1 & a_2 & \cdots & a_n\end{bmatrix}$ and a $n\times 1$ column vector, $B=\begin{bmatrix}b_1 \\ b_2 \\ \vdots \\ b_n\end{bmatrix}$. The matrix product is defined as $a_1b_1 + a_2b_2 + \cdots + a_nb_n$. This operation is not commutative.
\subsubsection*{Example}
\begin{itemize}
	\item $C * B$
\begin{align*}
	\begin{bmatrix}
		4 & 5 & 6
	\end{bmatrix} *
	\begin{bmatrix}
		4 \\
		0 \\
		6
	\end{bmatrix} =
	4*4 + 5*0 + 6*6 = 16+ 36 = 52
\end{align*}
\end{itemize}

\subsection{Matrix Product Existence and Dimension}
Recall that the dimensions of a matrix are given as $n\times m$, where n is the number of rows and m is the number of columns. In order for the matrix product to be defined, we must have that the \colorbox{yellow!50}{number of columns of the first} matrix is equal to the \colorbox{yellow!50}{number of rows of the second} matrix. When this is true, the product matrix has dimensions of the \colorbox{blue!20}{number of rows of the first} matrix by the \colorbox{blue!20}{number of columns of the second} matrix.
\subsubsection*{Examples}
For these examples, just use the dimensions of the given matrices in order.
\begin{itemize}
	\item $A * B$, $\bhighlight{3}\times \yhighlight{1}$ and $\yhighlight{3} \times \bhighlight{1}$. Notice that the inner terms, $1\neq 3$, therefore; this operation is not defined.
	\item $A * C$, $\bhighlight{3} \times \yhighlight{1}$ and  $\yhighlight{1}\times \bhighlight{3}$. Since the inner terms, $1 = 1$, the product exists and its dimension is $3\times 3$.
	\item $A * D$, $\bhighlight{3}\times \yhighlight{1}$ and $\yhighlight{2} \times \bhighlight{3}$. Notice that the inner terms, $1\neq 3$, therefore; this operation is not defined.
	\item $D * A$, $\bhighlight{2} \times \yhighlight{3}$ and  $\yhighlight{3}\times \bhighlight{1}$. Since the inner terms, $3 = 3$, the product exists and its dimension is $2\times 1$.
\end{itemize}

\subsection{Calculate the Product of Matrices}
Recall how we found the product of a row vector and a column vector. It is easier to understand how to find the product by example but please do read the definition of \textit{Matrix Product} on page 215. Let's find the product of $D*F$.
\\\\
D is $\bhighlight{2} \times \yhighlight{3}$ and  F is $\yhighlight{3}\times \bhighlight{2}$. Since the inner terms, $3 = 3$, the product exists and its dimension is $2\times 2$. (\textbf{always do this check first})
\begin{align*}
	\begin{bmatrix}
		\tikzmark{left1}1 & 2 & 3\tikzmark{right1} \\
		4 & 0 & 6
	\end{bmatrix} &\times
	\begin{bmatrix}
		\tikzmark{left2}3 & 3 \\
		4 & 1 \\
		2\tikzmark{right2} & 0
	\end{bmatrix} = 
	\begin{bmatrix}
		p_{1 1} & p_{1 2} \\
		p_{2 1} & p_{2 2}
	\end{bmatrix}
	\\
	p_{1 1} = \begin{bmatrix}1 & 2 & 3\end{bmatrix} &\times
	\begin{bmatrix}3 \\ 4 \\ 2\end{bmatrix} = 1*3 + 2*4 + 3*2 =	17
\end{align*}
\DrawBox[thick, red ]{left1}{right1}{}
\DrawBox[thick, blue ]{left2}{right2}{}\\
\begin{align*}
	\begin{bmatrix}
		\tikzmark{left1}1 & 2 & 3\tikzmark{right1} \\
		4 & 0 & 6
	\end{bmatrix} &\times
	\begin{bmatrix}
		3 & \tikzmark{left2}3 \\
		4 & 1 \\
		2 & 0\tikzmark{right2}
	\end{bmatrix}
	\\
	p_{1 2} = \begin{bmatrix}1 & 2 & 3\end{bmatrix} &\times
	\begin{bmatrix}3 \\ 1 \\ 0\end{bmatrix} = 3+2+0=5
\end{align*}
\DrawBox[thick, red ]{left1}{right1}{}
\DrawBox[thick, blue ]{left2}{right2}{}

\cleardoublepage

\begin{align*}
	\begin{bmatrix}
		1 & 2 & 3 \\
		\tikzmark{left1}4 & 0 & 6\tikzmark{right1}
	\end{bmatrix} &\times
	\begin{bmatrix}
		\tikzmark{left2}3 & 3 \\
		4 & 1 \\
		2\tikzmark{right2} & 0
	\end{bmatrix}
	\\
	p_{2 1} = \begin{bmatrix}4 & 0 & 6\end{bmatrix} &\times
	\begin{bmatrix}3 \\ 4 \\ 2\end{bmatrix} =12+0+12=24
\end{align*}
\DrawBox[thick, red ]{left1}{right1}{}
\DrawBox[thick, blue ]{left2}{right2}{}

\begin{align*}
	\begin{bmatrix}
		1 & 2 & 3 \\
		\tikzmark{left1}4 & 0 & 6\tikzmark{right1}
	\end{bmatrix} &\times
	\begin{bmatrix}
		3 & \tikzmark{left2}3 \\
		4 & 1 \\
		2 & 0\tikzmark{right2}
	\end{bmatrix}
	\\
	p_{2 2} = \begin{bmatrix}4 & 0 & 6\end{bmatrix} &\times
	\begin{bmatrix}3 \\ 1 \\ 0\end{bmatrix} = 12+0+0=12
\end{align*}
\DrawBox[thick, red ]{left1}{right1}{}
\DrawBox[thick, blue ]{left2}{right2}{}
So that
\begin{align*}
	D*F = \begin{bmatrix}
		p_{1 1} & p_{1 2} \\
		p_{2 1} & p_{2 2}
	\end{bmatrix} =
	\begin{bmatrix}
		17 & 5 \\
		24 & 12
	\end{bmatrix}
\end{align*}

\subsubsection*{Examples}
\begin{itemize}
	\item $D*E$ is $\bhighlight{2} \times \yhighlight{3}$ and $\yhighlight{3}\times \bhighlight{2}$. Since the inner terms, $3 = 3$, the product exists and its dimension is $2\times 2$.
\begin{align*}
	\begin{bmatrix}
		1 & 2 & 3 \\
		4 & 0 & 6
	\end{bmatrix} \times
	\begin{bmatrix}
		2 & 1 \\
		1 & 2 \\
		3 & 4
	\end{bmatrix} &= 
	\begin{bmatrix}
		1*2+2*1+3*3 & 1*1+2*2+3*4 \\
		4*2+0+6*3 & 4*1+0+6*4
	\end{bmatrix} \\
	& = \begin{bmatrix}
		13 & 17 \\
		24 & 28
	\end{bmatrix}
\end{align*}
\item $E*D$ is $\bhighlight{3} \times \yhighlight{2}$ and $\yhighlight{2}\times \bhighlight{3}$. Since the inner terms, $2 = 2$, the product exists and its dimension is $3\times 3$.
\begin{align*}
	\begin{bmatrix}
		2 & 1 \\
		1 & 2 \\
		3 & 4
	\end{bmatrix} \times
	\begin{bmatrix}
		1 & 2 & 3 \\
		4 & 0 & 6
	\end{bmatrix} 
	 &= 
	\begin{bmatrix}
		2*1+1*4 & 2*2+0 & 2*3+1*6 \\
		1*1+2*4 & 1*2+0 & 1*3+2*6 \\
		3*1+4*4 & 3*2+0 & 3*3+4*6
	\end{bmatrix} \\
	& = \begin{bmatrix}
		6 & 4 & 12 \\
		9 & 2 & 15 \\
		19 & 6 & 33
	\end{bmatrix}
	\end{align*}
\end{itemize}

\subsubsection*{Not Commutative,  But Associative, and Distributive}
Assuming the operation is defined:
\begin{itemize}
	\item Matrix products are \textbf{not commutative}, $D\times E$ may not equal $E\times D$. This is so even if the matrices are square.
	\item Matrix products are associative, $(F\times D) \times E = F\times (D\times E)$.
	\item Matrices are distributive, $D(E+F) = DE + DF$.
\end{itemize}

\subsection{Summary}
\begin{itemize}
	\item Always verify first that the matrix operation is defined by checking the dimensions of the input matrices.
	\item Determine the dimensions of the result matrix.
	\item Take your time performing matrix products. Row to column.
	\item Try to do all of the homework problems 33-48 on page 221.	
	\item $A^2=A\times A$. This is only defined if $A$ is a square matrix ($n=m$).
\end{itemize}

\noindent\rule{\textwidth}{1pt}
{\footnotesize Copyright (C) 2021 Garold Dalton --- Released under GNU General Public License v3.0}

\end{document}
