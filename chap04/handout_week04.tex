\documentclass[14pt]{extarticle}
\usepackage{amsmath}
\usepackage{amssymb}
\usepackage{graphicx}
%\usepackage{tikz}
%\usetikzlibrary{calc}
%\usetikzlibrary{trees}
\usepackage{graphicx}
\graphicspath{ {../} }
\usepackage[top=0.75in, bottom=0.75in, left=0.75in, right=0.75in]{geometry}
\newcommand*{\Scale}[2][4]{\scalebox{#1}{\ensuremath{#2}}}%
\usepackage[shortlabels]{enumitem}
% \usepackage{showframe}
\title{\vspace{-5ex}HANDOUT Math 208 Week 04}
\date{\vspace{-10ex}}
\usepackage{multicol}
\setlength{\columnsep}{1cm}

\begin{document}
\maketitle	
\pagenumbering{gobble}	
\section*{Chapter 3 Review}
\begin{itemize}
	\item I wish to compare investments but Investment A is continuous compounding and Investment B is compounded semi-annually. What formulas do I use?
	\item What is the interest rate if my investment triples after 14 years of continuous compounding?
	\item What is $m$ given compounding of: a) Quarterly, b) Monthly, c) Semi-annually?
	\item  Charges on a payday loan may not exceed \$15.50 per \$100 borrowed. Find the annual interest rate if \$400 is borrowed for 13 days at the maximum allowable charge.
\end{itemize}

\section*{Chapter 4 Review}
\begin{enumerate}
	\item \begin{itemize}
		\item What is the coefficient matrix, A? Is A singular? What is $A^{-1}$?
		\item Solve by substitution, augmented matrix, and matrix equation.
	\end{itemize}
	\begin{align*}
		\text{Given: } &x+2y = 5\\
		&3x - 10y =2	
	\end{align*}

	\item Give examples of two variable systems, in augmented matrix form, with a) unique solution, b) infinite solutions, c) no solution. If your example has a solution, what is it?
	
	\item Find x, y, and z
	Given
	$$\begin{bmatrix}
		-x + 5y & z \\
		-3 & 3x + 2y
	\end{bmatrix}
	= 
	\begin{bmatrix}
		2 & -9 \\
		-3 & 11
	\end{bmatrix}
	$$
	
	\item Set up the augmented matrices to find a) the solution and b) the inverse of the coefficient matrix. Given:
	\begin{align*}
		\text{Given: } &2x_1+6x_2 + 15x_3 = -12 \\
		&4x_1+7x_2 + 13x_3= -10 \\
		&3x_1+6x_2 + 12x_3 = -9
	\end{align*}
	Know how to find these using Gauss-Jordan.
	
	\item What are the following matrices in reduce form? What are the solutions?
	\begin{multicols}{3}
		\begin{align*}
			&\begin{bmatrix}
				1 & 0 & | & 1\\
				0 & 5 & | & 2
			\end{bmatrix} 
		\end{align*}
		\\
		\begin{align*}
			&\begin{bmatrix}
				1 & 0 & 0 & | & 3\\
				0 & 1 & 0 & | & 2 \\
				0 & 0 & 1 & | & -5
			\end{bmatrix} 
		\end{align*}
		\vfill\null
		\columnbreak
		\begin{align*}
			&\begin{bmatrix}
				1 & 9 & | & 3\\
				0 & 0 & | & 0
			\end{bmatrix} 
		\end{align*}
		\\
		\begin{align*}
			&\begin{bmatrix}
				1 & 7 & 0 & | & -2\\
				0 & 1 & 2 & | & 3 \\
				0 & 0 & 0 & | & 0
			\end{bmatrix} 
		\end{align*}
		\vfill\null
		\columnbreak
		\begin{align*}
			&\begin{bmatrix}
				-1 & 2 & | & 2\\
				0 & 0 & | & k
			\end{bmatrix} 
		\end{align*}
		\\
		\begin{align*}
			&\begin{bmatrix}
				1 & 3 & 7 & | & h\\
				0 & 0 & 0 & | & k \\
				0 & 0 & 0 & | & 0
			\end{bmatrix} 
		\end{align*}
		\vfill\null
	\end{multicols}
	
	\item Given:
	\begin{align*}
		&A = \begin{bmatrix}
			1 \\
			2 \\
			3
		\end{bmatrix} &
		&B = \begin{bmatrix}
			4 \\
			0 \\
			6
		\end{bmatrix} &
		&C =\begin{bmatrix}
			4 & 5 & 6
		\end{bmatrix} \\
		&D = \begin{bmatrix}
			1 & 2 & 3 \\
			4 & 0 & 6
		\end{bmatrix} &
		&E = \begin{bmatrix}
			2 & 1 \\
			1 & 2 \\
			3 & 4
		\end{bmatrix} &
		&F = \begin{bmatrix}
			3 & 3 \\
			4 & 1 \\
			2 & 0
		\end{bmatrix}
	\end{align*}
	Find:
	\begin{multicols}{3}
		\begin{align*}
			&A+B = \\
			&A+C = \\
			&2E-D = \\
			&2E-F =
		\end{align*}
		\vfill\null
		\columnbreak
		\begin{align*}
			&AB = \\
			&AC = \\
			&DE = \\
			&EF =
		\end{align*}
		\vfill\null
		\columnbreak
		\begin{align*}
			&AI = \\
			&A^2 = \\
			&I^3D = \\
			&AC(AC)^{-1} =
		\end{align*}
		\vfill\null
	\end{multicols}	
\end{enumerate}


\section*{Chapter 5}
\begin{enumerate}
	\item Find two points on the lines, a) $2x-3y=6$ and b) $2y = 4x-10$.
	\item Is the origin in the half-plane of  a) $2x-3y>6$ and b) $2y \geq 4x-10$.
	\item Graph these inequalities.
\end{enumerate}


\end{document}
